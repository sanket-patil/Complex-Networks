% ****** Start of file apssamp.tex ******
%
%   This file is part of the APS files in the REVTeX 4 distribution.
%   Version 4.0 of REVTeX, August 2001
%
%   Copyright (c) 2001 The American Physical Society.
%
%   See the REVTeX 4 README file for restrictions and more information.
%
% TeX'ing this file requires that you have AMS-LaTeX 2.0 installed
% as well as the rest of the prerequisites for REVTeX 4.0
%
% See the REVTeX 4 README file
% It also requires running BibTeX. The commands are as follows:
%
%  1)  latex apssamp.tex
%  2)  bibtex apssamp
%  3)  latex apssamp.tex
%  4)  latex apssamp.tex
%
\documentclass[twocolumn,showpacs,preprintnumbers,amsmath,amssymb]{revtex4}
%\documentclass[preprint,showpacs,preprintnumbers,amsmath,amssymb]{revtex4}

% Some other (several out of many) possibilities
%\documentclass[preprint,aps]{revtex4}
%\documentclass[preprint,aps,draft]{revtex4}
%\documentclass[prb]{revtex4}% Physical Review B

\usepackage{graphicx}% Include figure files
\usepackage{dcolumn}% Align table columns on decimal point
\usepackage{bm}% bold math

%\nofiles

\begin{document}

\preprint{APS/123-QED}

\title{Characterization of Efficiency-Robustness Trade-offs in Complex Networks}% Force line breaks with \\

\author{Ann  Author}
 \altaffiliation[Also at ]{Physics Department, XYZ University.}%Lines break automatically or can be forced with \\
\author{Second Author}%
 \email{Second.Author@institution.edu}
\affiliation{%
Authors' institution and/or address
}%

\author{Charlie Author}
 \homepage{http://www.Second.institution.edu/~Charlie.Author}
\affiliation{
Second institution and/or address% with \\
}%

\date{\today}% It is always \today, today,
             %  but any date may be explicitly specified

%\begin{abstract}
%We present a study of efficiency and robustness of several real world complex networks. We report a detailed analysis of the topologies of major US domestic airlines networks. We also analyze other classes of complex networks, such as food-webs, trade networks and supply-chains. Interesting structural motifs can be observed both within and across classes of networks. Some of these motifs are potential design principles for complex networks.
%\end{abstract}

\pacs{Valid PACS appear here}% PACS, the Physics and Astronomy
                             % Classification Scheme.
%\keywords{Suggested keywords}%Use showkeys class option if keyword
                              %display desired
\maketitle

\section{\label{sec:intro}Introduction}

%Complex Networks are characterized by non-trivial structural properties. 
There is a lot of research interest in understanding the structural properties of complex networks, especially following the seminal work by Strogatz, Albert and Barab\`{a}si~\cite{strogatz01, albert02, barabasi03}. Classes of complex networks have been studied in search of specific optimal properties such as diameter~\cite{barabasi99}, degree distribution~\cite{albert02}, clustering coefficient~\cite{watts98} and modularity~\cite{newman}. Various real world complex networks have been studied in literature, such as metbolic networks~\cite{guimera07, smart08}, airline networks~\cite{guimera04, guimera05}, food webs~\cite{williams98, cochen}, protein interaction networks[citation needed], social networks~\cite{newman} and trade networks~\cite{vespignani}. %Unlike random graph models, complex networks have either \textit{evolved} over time for survival or have been \textit{designed} for performance. They are teleological in the sense that each complex network has (or is built to serve) a purpose.

The underlying structure or topology impacts not only the performance of a complex network, but also its survival in the phase of environmental uncertainties. Hence, a detailed understanding of the structures of complex networks is warranted. In this work, we analyze different classes of complex networks and study their structural properties with respect to two parameters critical to network performance: \textit{efficiency} and \textit{robustness}. [Need to say something about results here.]

A large body of literature exists that attempts to characterize complex networks inspired from ideas in statistical physics such as \textit{emergence} and \textit{self-organization}. The most popular model of self-organization is \textit{self-organized criticality} (SOC) proposed by Bak~\cite{soc}. %and \textit{edge of chaos} introduced by Kauffman~\cite{eoc}.
The essential idea behind SOC is that systems with a large number of interacting components will ``self-organize'' to a critical point at which events follow scale-invariance leading to power laws. Since this behaviour ``emerges'' without the need for tuning parameters, system design has only a minor role. %The intuition behind EOC is that complex systems lie on a single  

Network generation models developed in efforts to explain seemingly universal motifs in complex networks such as scale-free degree distribution, community structures and small diameters use ideas similar to SOC. Barab\`{a}si and Albert~\cite{barabasi} propose \textit{preferential attachment} as the underlying mechanism for the emrgence of power laws in nature and man-made networks. Watts and Strogatz~\cite{watts} propose a model to generate networks that exhibit ``small-world'' properties with small average path lengths and high clustering coefficients leading to the formation of ``communities''. The Watts-Strogatz model uses a parameter $\beta$ that determines the amount of randomness in the network. Kleinberg~\cite{kleinberg} provides a theoretical framework for small-world networks by developing exact conditions under which small-worlds occur. Newman~\cite{newman} reports the existence of community structures in disparate complex networks.

In contrast to the statistical mechanical explanations of phenomema like power laws and scale-invariance as emerging due to simple underlying forces such as randomness and preferential attachment, there are recent efforts which propose that the occurrence of complex phenomena are a result of meticulous \textit{design} for functionality and performance. A system's functional requirements are defined in terms optimality objectives. The occurrence of interesting motifs is thus the result of trade-offs between objectives. Alderson and Willinger~\cite{hot} propose \textit{highly optimized tolerance} (HOT) as a mechanism to generate power laws as a result of trade-offs between performance and robustness in a network. They also challenge the validity of models such as scale-free networks since such models ignore domain dependent trade-offs~\cite{hotnets}. Further, the scale-free model attempts to characterize networks based solely on degree distribution, which might be inadequate~\cite{hot-mathematics09}.

Venkatasubramanian et al.~\cite{venkat04} study the emergence of network topologies under different efficiency and robustness constraints. They propose a formalism based on the Darwinian evolutionary theory. Efficiency indicates the average case or short term survival of a complex network; whereas robustness indicates the worst case or the long term survival. A selection pressure variable decides the trade-off between efficiency and robustness. Using these measures, they let topologies evolve under different trade-offs through a genetic algorithm process. Efficiency is defined in terms of the average path length of the resulting graph topology and robustness in terms of the number (and size) of connected components a node deletion causes in the graph. The star topology emrges as most efficient and least robust, when the selection pressure is completely on efficiency. The circle topology emerges as the most robust and least efficient, when the selection pressure is completely on robustness. Hub and spoke structures of diffrent types emerge for intermediate selection pressures.

We extend this formalism to analyze the efficieny and robustness of several classes of complex networks such as airline networks, food-webs and supply-chains. We use multiple measures of efficiency and robustness that characterize the different aspects of optimality in networks. We observe optimal motifs within classes of networks as well as across classes. We conjecture that these optimal motifs are potential underpinnings for the design of complex networks. [Need to be more specific.] [Should mention important results here.]

Trade-offs have been studied in the context of several real world complex networks. Alderson \textit{et al.} study the trade-offs at the router level as well as the autonomous system level in the internet~\cite{hot-internet}. Amaral \textit{et al.} analyze the world wide air transportation network~\cite{amaral05} and report an anamolous betweenness distribution, as a result of geo-political trade-offs. Martinez \textit{et al.} study foodwebs extensively and demonstrate their \textit{connectance} and degree distribution. William and Martinez develop a simple model of foodwebs~\cite{foodweb-model}.
[Need to add much more literature.]

[Need to do better positioning here.]
%Generally, the work in characterizing complex networks has taken into consideration a only a small number of properties such as degree distribution, betweenness and clustering. Performance of a network has multiple aspects. As such, we present a mor
%We study the individual US domestic airlines networks in detail to show that they have similar design trade-offs. Similarly, we study properties of food-webs and supply-chains.



%We begin by introducing the graph theoretic formalism (section \ref{sec:formalism}). We briefly discuss the different complex networks that we studied (section \ref{sec:networks}). We then present a thorough analysis of the efficiency and robustness of each class of complex networks (section \ref{sec:analysis}). We present a philosophical discussion of interesting design principles in section \ref{sec:discussion}. Finally, we position this work with respect to existing literature and indicate future directions \ref{sec:rellit}.

\section{preliminaries}

%Networks are modelled as graphs. In the rest of the paper, the terms of graph and network are used interchangeably to mean the same.
%
%A graph, $G(V, E)$ is a set of of vertices (or nodes) $V = \{1, 2, ..., n\}$ and a set of edges (or links) $E = \{e_{ij}: i \in V\ and\ j \in V\}$. A node can represent a machine, a human or a cell in a network. An edge can represent a \textit{relation} between two nodes. For example, in a social network, an edge represents an ``acquaintance'' relation. In case of a food web an edge is a ``predator-prey'' relation. An edge can also represent \textit{flow} of material and/or information as in supply chains and computer networks.
%
%Edges can be \textit{undirected} (or bidirected) or \textit{directed}, indicating the direction of a relation or flow. Edge \textit{weights} are used to convey additional (extra-topological) information. For example: in an airline network, weight on an edge can be distance, or estimated flight time.
%
%\textit{Degree} is the number of edges incident on a node. If the edges are directed, then we measure \textit{indegree} (the number of incoming edges) and \textit{outdegree} (the number of outgoing edges) of a node. The degree sequence of a graph is the sequence of node degrees. The degree distribution, $P(k)$, of a graph is the fraction of the nodes in the graph, $n$, that have a degree $k$.

%A \textit{path} in a graph is a sequence of edges from a source node to a destination node. The number of edges in a path is called its \textit{pathlength}. If the edges have weights on them, then the pathlength of a path is the  sum of the weights. A \textit{shortest path} between a source and a destination is a path between the two nodes that has the smallest pathlength.
%
%The \textit{average path length} (APL) of a graph is computed by computing the average of the shortest pathlengths between all pairs of nodes in the graph. The longest of the all pairs shortest paths is called a diameter path, and its pathlength is called the \textit{diameter} of the graph.
%
%While the diameter is a property of a network, similar definitions at the level of individual nodes are also useful. For a node, the longest pathlength among its shortest paths is called the node's \textit{eccentricity}. It is the greatest separation of a node from any other node in the network. Diameter is the highest eccentricity across all nodes in the network. Similarly, the smallest eccentricity across all nodes in the network is called the \textit{radius} of the network.
%
%Various measures of \textit{centrality} of nodes in a network are used to determine the relative ``importance'' of the nodes within a network. Below we define some of the important ones.
%
%\textit{Degree Centrality} is usually the same as the degree sequence. However, it is often represented in a normalized measure. We normalize the degree centrality of a node against the total degree in the graph, $\sum_{i}k_{i}$, where $k_{i}$ is the degree of $i$. The normalized degree centrality of a node $i$, $C_{D}(i)$, is defined as -
%\[C_{D}(i) = \frac{k_{i}}{\sum_{i}k_{i}}\]
%
%\textit{Betweenness Centrality} of a node is measured in terms of the number of times the node occurs in shortest paths as an intermediate node. For a node $i$, and for every node pair $(j, k)$ in a graph, the betweenness centrality of $i$, $C_{B}(i)$, is defined as -
%\[C_{B}(i) = \sum_{j \ne i \ne k, j \ne k}\frac{\sigma_{jk}^{i}}{\sigma_{jk}}\]
%
%Here, $\sigma_{jk}$, is the number of shortest paths between $j$ and $k$. and $\sigma_{jk}^{i}$, is the number of shortest paths between $j$ and $k$ via $i$.
%
%\textit{Betweenness Centrality} of an edge is measured in terms of the number of times the edge occurs in shortest paths. For an edge $e_{ij}$, and for every node pair $(k, l)$ in a graph, the betweenness centrality of $e_{ij}$, $C_{B}(e_{ij})$, is defined as - 
%\[C_{B}(e_{ij}) = \sum_{k \ne l}\frac{\sigma_{kl}^{e_{ij}}}{\sigma_{kl}}\]
%
%\textit{Closeness Centrality} of a node has the connotation of the average path length of that node. In other words, it is the average distance from the node to any other node. For a node $i$, the closeness centrality, $C_{cl}(i)$, is defined as -
%
%\[C_{cl}(i) = \frac{\sum_{j, j \ne i} pl(i, j)}{n - 1}\]
%
%Here, $pl(i, j)$ is the pathlength between $i$ and $j$.
%
%Often, the performance of a network depends on how well the network is connected. A network is said to be \textit{strongly connected} if there exists a path from any node to any other node in the network. A network is said to be \textit{disconncted} otherwise. Since node and edge failures can happen in a network, it is important to measure the resilience of the network in the face of failures. The \textit{vertex cut} of a network is a set of vertices, whose removal renders the network disconnected. The \textit{minimal vertex cut} is the smallest of such vertex cuts: it is the minimum number of vertices whose removal disconnects a network. Similarly, \textit{edge cut} and \textit{minimal edge} cut are defined.
%
%The vertex or node connectivity of a network is the size of the minimal vertex cut. The edge connectivity of a network is the size of the minimum edge cut. There is a well known theorem by Menger which states that the maximum number of \textit{independent paths} in a network is equal to the size of the minimal cut. A set of paths is said to be \textit{vertex independent}, if there is no shared vertex. Similarly, a set of paths is \textit{edge independent}, if there is no shared edge. 

\subsection{Critical Parameters}

\subsubsection{Efficiency}
Efficiency is indicative of the short-term survival of a network. Survival of a network entails \textit{communication} of a node with other nodes. A certain cost is associated with communication. Efficiency measures the cost of \textit{communication} in a network. There are a number of ways in which efficiency can be defined. The \textit{diameter} of a network is the upper bound on the communication cost in the network. The \textit{average path length} (APL) gives the communication cost on average. Maximizing the symmetry in the distribution of distances between pairs of nodes can also serve as a useful measure of efficiency. Therefore, per node \textit{eccentricity} (longest of all shortest paths from a node) and \textit{closeness centrality} values can also be used to define efficiency.

The worst diameter for a connected graph of $n$ nodes is $n - 1$, which is the diameter of a straight line graph, in case of undirected graphs, and a circle in case of directed graphs. The best diameter is $1$, which is the diameter of a clique (complete graph). In other words, a topology is most efficient if the diameter is $1$, and least efficient if it is  $n - 1$. We map a diameter \textit{d} that falls in the interval $\left[1, n - 1\right]$, to a value of efficiency in the interval $\left[0, 1\right]$, as:

\[ \eta = 1 - \frac{d - 1} {n - 2} \]

By this definition, a straight line topology has an $\eta$ of 0, a circular topology has an $\eta$ nearing 0.5, a clique has an $\eta$ of 1 and so on.

The worst APL for a connected undirected graph of $n$ nodes, which occurs again for a straight line, is $\frac{n + 1}{3}$. Similarly, for directed graphs, the worst case APL, the APL of the circle, is $\frac{n}{2}$. The best case APL in either case, is $1$, for the clique. Thus, when measured in terms of APL, efficiency:

\[\eta = 1 - 3\frac{apl - 1} {n - 2},\ for\ undirected\ graphs\]
\[\eta = 1 - 2\frac{apl - 1} {n - 2},\ for\ directed\ graphs\]

\subsubsection{Robustness}
Robustness measures the resilience of a network in the face of node and edge failures. Robustness is often defined in terms of the \textit{skew} in the \textit{importance} of nodes and edges. As such, centrality measures viz; degree, node betweenness and edge betweenness distributions can be used. When there is a skew in the centrality measures, a small number of nodes and/or edges are more important than the others. Thus, their failure affects the network's performance much more than failures in the rest of the network. On the other hand, a symmetric centrality distribution ensures robustness to random as well as targetted node/edge failures.

One of the definitions of robustness we use is skew in degree centrality. We define this as the difference in the maximum degree in the graph ($\hat{p}$) and the mean degree of the nodes ($\bar{p}$). For a connected graph of $n$ nodes, the worst skew occurs for the star topology. The central node has a degree of $n - 1$ and all the nodes surrounding it have a degree of 1. Therefore, the worst skew is $\frac{(n - 1)(n - 2)}{n}$. The best skew is 0, when all the nodes have the same degree. This occurs when the topologies are \textit{regular} graph topologies as in a circular topology or a clique. This holds for both directed and undirected graphs. Thus,

\[\rho = 1 - \frac{n(\hat{p} - \bar{p})}{(n - 1)(n - 2)}\]

Another way to measure robustness is in terms of \textit{betweenness}. 

Another way to measure robustness is in terms of \textit{connectivity} ($\lambda$). Connectivity is the minimum number of nodes or edges whose removal renders the network disconnected. In case of an undirected graph, the tree topologies have the worst connectivity of $1$, and the circle has the worst connectivity of $1$ in directed graphs. For both cases, the clique has the best connectivity, $n - 1$. Thus, robustness, when defined in terms of connectivity is:
\[\rho = \frac{\lambda - 1}{n - 2}\]

\subsubsection{Cost}
The minimum number of edges ($e_{min}$) required to have a connected undirected graph is $n - 1$ and $n$ is case of a directed graph. We do not associate any cost to a minimally connected graph. Any ``extra'' edge has an associated cost. All extra edges cost the same. An undirected clique has the highest cost, with $\hat{e} = \frac{n(n - 1)}{2}$ (and $\hat{e} = n(n - 1)$, for directed) number of edges. Thus, the cost of a topology is defined as a ratio of the number of extra edges in a topology to the number of extra edges in the clique with the same number of nodes.

\[ k = \frac{e - e_{min}} {\hat{e} - e_{min}} \]

In case of weighted graphs, cost is measured in terms of the total weight over edges.

\subsubsection{Selection Pressure Estimate}
Venkatasubramanian et al.~\cite{venkat04} propose the selection pressure variable $\alpha$ that decides the trade-off between efficiency and robustness during the design process of a complex network. In this work, we propose a complementary measure $\beta$ which is an \textit{estimate} of $\alpha$ that might have been used to design a particular network topology.

\section{Classes of Complex Networks}
\subsection{Foodwebs}
\subsection{Supply Chain Networks}
\subsection{Airline Networks}

\section{Performance Analyses}


%\subsection{Efficiency Analyses}
%\subsection{Robustness Analyses}
%\subsection{Comparison with Random Graphs}
%\subsection{Interesting Design Motifs}

\section{Related Literature}

\section{Future Work}
%* What are complex networks: (1) man made (2) natural\\
%
%* Special structural properties -- structure influences function\\
%
%* Efficiency, robustness and cost as three principal dimensions.\\
%
%* In this work, we study: (1) structural properties (2) subject a variety of complex n/ws to graph theoretic analysis (2) wrt eff, rob and cost.\\


%\section{Related Literature}
%Subsection text here.
%
%
%
%\subsection{Efficiency measures}
%* APL\\
%* Diameter\\
%
%\subsection{Robustness measures}
%* SECON measures\\
%* Degree distribution based\\
%* Connectivity based\\
%* Centrality based measures\\
%
%\subsection{Cost measures}
%* Redundancy\\
%* Density\\
%
%\section{Efficiency and Robustness Analyses}
%
%\section{Discussion}


%When commands are referred to in this example file, they are always
%shown with their required arguments, using normal \TeX{} format. In
%this format, \verb+#1+, \verb+#2+, etc. stand for required
%author-supplied arguments to commands. For example, in
%\verb+\section{#1}+ the \verb+#1+ stands for the title text of the
%author's section heading, and in \verb+\title{#1}+ the \verb+#1+
%stands for the title text of the paper.
%
%Line breaks in section headings at all levels can be introduced using
%\textbackslash\textbackslash. A blank input line tells \TeX\ that the
%paragraph has ended. Note that top-level section headings are
%automatically uppercased. If a specific letter or word should appear in
%lowercase instead, you must escape it using \verb+\lowercase{#1}+ as
%in the word ``via'' above.

%\subsection{\label{sec:level2}Second-level heading: Formatting}
%
%This file may be formatted in both the \texttt{preprint} and
%\texttt{twocolumn} styles. \texttt{twocolumn} format may be used to
%mimic final journal output. Either format may be used for submission
%purposes; however, for peer review and production, APS will format the
%article using the \texttt{preprint} class option. Hence, it is
%essential that authors check that their manuscripts format acceptably
%under \texttt{preprint}. Manuscripts submitted to APS that do not
%format correctly under the \texttt{preprint} option may be delayed in
%both the editorial and production processes.
%
%The \texttt{widetext} environment will make the text the width of the
%full page, as on page~\pageref{eq:wideeq}. (Note the use the
%\verb+\pageref{#1}+ to get the page number right automatically.) The
%width-changing commands only take effect in \texttt{twocolumn}
%formatting. It has no effect if \texttt{preprint} formatting is chosen
%instead.
%
%\subsubsection{\label{sec:level3}Third-level heading: References and Footnotes}
%Reference citations in text use the commands \verb+\cite{#1}+ or
%\verb+\onlinecite{#1}+. \verb+#1+ may contain letters and numbers.
%The reference itself is specified by a \verb+\bibitem{#1}+ command
%with the same argument as the \verb+\cite{#1}+ command.
%\verb+\bibitem{#1}+ commands may be crafted by hand or, preferably,
%generated by using Bib\TeX. REV\TeX~4 includes Bib\TeX\ style files
%\verb+apsrev.bst+ and \verb+apsrmp.bst+ appropriate for
%\textit{Physical Review} and \textit{Reviews of Modern Physics},
%respectively. REV\TeX~4 will automatically choose the style
%appropriate for the journal specified in the document class
%options. This sample file demonstrates the basic use of Bib\TeX\
%through the use of \verb+\bibliography+ command which references the
%\verb+assamp.bib+ file. Running Bib\TeX\ (typically \texttt{bibtex
%apssamp}) after the first pass of \LaTeX\ produces the file
%\verb+apssamp.bbl+ which contains the automatically formatted
%\verb+\bibitem+ commands (including extra markup information via
%\verb+\bibinfo+ commands). If not using Bib\TeX, the
%\verb+thebibiliography+ environment should be used instead.
%
%To cite bibliography entries, use the \verb+\cite{#1}+ command. Most
%journal styles will display the corresponding number(s) in square
%brackets: \cite{feyn54,witten2001}. To avoid the square brackets, use
%\verb+\onlinecite{#1}+: Refs.~\onlinecite{feyn54} and
%\onlinecite{witten2001}. REV\TeX\ ``collapses'' lists of
%consecutive reference numbers where possible. We now cite everyone
%together \cite{feyn54,witten2001,epr}, and once again
%(Refs.~\onlinecite{epr,feyn54,witten2001}). Note that the references
%were also sorted into the correct numerical order as well.
%
%When the \verb+prb+ class option is used, the \verb+\cite{#1}+ command
%displays the reference's number as a superscript rather than using
%square brackets. Note that the location of the \verb+\cite{#1}+
%command should be adjusted for the reference style: the superscript
%references in \verb+prb+ style must appear after punctuation;
%otherwise the reference must appear before any punctuation. This
%sample was written for the regular (non-\texttt{prb}) citation style.
%The command \verb+\onlinecite{#1}+ in the \texttt{prb} style also
%displays the reference on the baseline.
%
%Footnotes are produced using the \verb+\footnote{#1}+ command. Most
%APS journal styles put footnotes into the bibliography. REV\TeX~4 does
%this as well, but instead of interleaving the footnotes with the
%references, they are listed at the end of the references\footnote{This
%may be improved in future versions of REV\TeX.}. Because the correct
%numbering of the footnotes must occur after the numbering of the
%references, an extra pass of \LaTeX\ is required in order to get the
%numbering correct.

%\section{Math and Equations}
%Inline math may be typeset using the \verb+$+ delimiters. Bold math
%symbols may be achieved using the \verb+bm+ package and the
%\verb+\bm{#1}+ command it supplies. For instance, a bold $\alpha$ can
%be typeset as \verb+$\bm{\alpha}$+ giving $\bm{\alpha}$. Fraktur and
%Blackboard (or open face or double struck) characters should be
%typeset using the \verb+\mathfrak{#1}+ and \verb+\mathbb{#1}+ commands
%respectively. Both are supplied by the \texttt{amssymb} package. For
%example, \verb+$\mathbb{R}$+ gives $\mathbb{R}$ and
%\verb+$\mathfrak{G}$+ gives $\mathfrak{G}$
%
%In \LaTeX\ there are many different ways to display equations, and a
%few preferred ways are noted below. Displayed math will center by
%default. Use the class option \verb+fleqn+ to flush equations left.
%
%Below we have numbered single-line equations; this is the most common
%type of equation in \textit{Physical Review}:
%\begin{eqnarray}
%\chi_+(p)\alt{\bf [}2|{\bf p}|(|{\bf p}|+p_z){\bf ]}^{-1/2}
%\left(
%\begin{array}{c}
%|{\bf p}|+p_z\\
%px+ip_y
%\end{array}\right)\;,
%\\
%\left\{%
% \openone234567890abc123\alpha\beta\gamma\delta1234556\alpha\beta
% \frac{1\sum^{a}_{b}}{A^2}%
%\right\}%
%\label{eq:one}.
%\end{eqnarray}
%Note the open one in Eq.~(\ref{eq:one}).
%
%Not all numbered equations will fit within a narrow column this
%way. The equation number will move down automatically if it cannot fit
%on the same line with a one-line equation:
%\begin{equation}
%\left\{
% ab12345678abc123456abcdef\alpha\beta\gamma\delta1234556\alpha\beta
% \frac{1\sum^{a}_{b}}{A^2}%
%\right\}.
%\end{equation}
%
%When the \verb+\label{#1}+ command is used [cf. input for
%Eq.~(\ref{eq:one})], the equation can be referred to in text without
%knowing the equation number that \TeX\ will assign to it. Just
%use \verb+\ref{#1}+, where \verb+#1+ is the same name that used in
%the \verb+\label{#1}+ command.
%
%Unnumbered single-line equations can be typeset
%using the \verb+\[+, \verb+\]+ format:
%\[g^+g^+ \rightarrow g^+g^+g^+g^+ \dots ~,~~q^+q^+\rightarrow
%q^+g^+g^+ \dots ~. \]
%
%\subsection{Multiline equations}
%
%Multiline equations are obtained by using the \verb+eqnarray+
%environment.  Use the \verb+\nonumber+ command at the end of each line
%to avoid assigning a number:
%\begin{eqnarray}
%{\cal M}=&&ig_Z^2(4E_1E_2)^{1/2}(l_i^2)^{-1}
%\delta_{\sigma_1,-\sigma_2}
%(g_{\sigma_2}^e)^2\chi_{-\sigma_2}(p_2)\nonumber\\
%&&\times
%[\epsilon_jl_i\epsilon_i]_{\sigma_1}\chi_{\sigma_1}(p_1),
%\end{eqnarray}
%\begin{eqnarray}
%\sum \vert M^{\text{viol}}_g \vert ^2&=&g^{2n-4}_S(Q^2)~N^{n-2}
%        (N^2-1)\nonumber \\
% & &\times \left( \sum_{i<j}\right)
%  \sum_{\text{perm}}
% \frac{1}{S_{12}}
% \frac{1}{S_{12}}
% \sum_\tau c^f_\tau~.
%\end{eqnarray}
%\textbf{Note:} Do not use \verb+\label{#1}+ on a line of a multiline
%equation if \verb+\nonumber+ is also used on that line. Incorrect
%cross-referencing will result. Notice the use \verb+\text{#1}+ for
%using a Roman font within a math environment.
%
%To set a multiline equation without \emph{any} equation
%numbers, use the \verb+\begin{eqnarray*}+,
%\verb+\end{eqnarray*}+ format:
%\begin{eqnarray*}
%\sum \vert M^{\text{viol}}_g \vert ^2&=&g^{2n-4}_S(Q^2)~N^{n-2}
%        (N^2-1)\\
% & &\times \left( \sum_{i<j}\right)
% \left(
%  \sum_{\text{perm}}\frac{1}{S_{12}S_{23}S_{n1}}
% \right)
% \frac{1}{S_{12}}~.
%\end{eqnarray*}
%To obtain numbers not normally produced by the automatic numbering,
%use the \verb+\tag{#1}+ command, where \verb+#1+ is the desired
%equation number. For example, to get an equation number of
%(\ref{eq:mynum}),
%\begin{equation}
%g^+g^+ \rightarrow g^+g^+g^+g^+ \dots ~,~~q^+q^+\rightarrow
%q^+g^+g^+ \dots ~. \tag{2.6$'$}\label{eq:mynum}
%\end{equation}
%
%A few notes on \verb=\tag{#1}=. \verb+\tag{#1}+ requires
%\texttt{amsmath}. The \verb+\tag{#1}+ must come before the
%\verb+\label{#1}+, if any. The numbering set with \verb+\tag{#1}+ is
%\textit{transparent} to the automatic numbering in REV\TeX{};
%therefore, the number must be known ahead of time, and it must be
%manually adjusted if other equations are added. \verb+\tag{#1}+ works
%with both single-line and multiline equations. \verb+\tag{#1}+ should
%only be used in exceptional case - do not use it to number all
%equations in a paper.
%
%Enclosing single-line and multiline equations in
%\verb+\begin{subequations}+ and \verb+\end{subequations}+ will produce
%a set of equations that are ``numbered'' with letters, as shown in
%Eqs.~(\ref{subeq:1}) and (\ref{subeq:2}) below:
%\begin{subequations}
%\label{eq:whole}
%\begin{equation}
%\left\{
% abc123456abcdef\alpha\beta\gamma\delta1234556\alpha\beta
% \frac{1\sum^{a}_{b}}{A^2}
%\right\},\label{subeq:1}
%\end{equation}
%\begin{eqnarray}
%{\cal M}=&&ig_Z^2(4E_1E_2)^{1/2}(l_i^2)^{-1}
%(g_{\sigma_2}^e)^2\chi_{-\sigma_2}(p_2)\nonumber\\
%&&\times
%[\epsilon_i]_{\sigma_1}\chi_{\sigma_1}(p_1).\label{subeq:2}
%\end{eqnarray}
%\end{subequations}
%Putting a \verb+\label{#1}+ command right after the
%\verb+\begin{subequations}+, allows one to
%reference all the equations in a subequations environment. For
%example, the equations in the preceding subequations environment were
%Eqs.~(\ref{eq:whole}).
%
%\subsubsection{Wide equations}
%The equation that follows is set in a wide format, i.e., it spans
%across the full page. The wide format is reserved for long equations
%that cannot be easily broken into four lines or less:
%\begin{widetext}
%\begin{equation}
%{\cal R}^{(\text{d})}=
% g_{\sigma_2}^e
% \left(
%   \frac{[\Gamma^Z(3,21)]_{\sigma_1}}{Q_{12}^2-M_W^2}
%  +\frac{[\Gamma^Z(13,2)]_{\sigma_1}}{Q_{13}^2-M_W^2}
% \right)
% + x_WQ_e
% \left(
%   \frac{[\Gamma^\gamma(3,21)]_{\sigma_1}}{Q_{12}^2-M_W^2}
%  +\frac{[\Gamma^\gamma(13,2)]_{\sigma_1}}{Q_{13}^2-M_W^2}
% \right)\;. \label{eq:wideeq}
%\end{equation}
%\end{widetext}
%This is typed to show the output is in wide format.
%(Since there is no input line between \verb+\equation+ and
%this paragraph, there is no paragraph indent for this paragraph.)
%\section{Cross-referencing}
%REV\TeX{} will automatically number sections, equations, figure
%captions, and tables. In order to reference them in text, use the
%\verb+\label{#1}+ and \verb+\ref{#1}+ commands. To reference a
%particular page, use the \verb+\pageref{#1}+ command.
%
%The \verb+\label{#1}+ should appear in a section heading, within an
%equation, or in a table or figure caption. The \verb+\ref{#1}+ command
%is used in the text where the citation is to be displayed.  Some
%examples: Section~\ref{sec:level1} on page~\pageref{sec:level1},
%Table~\ref{tab:table1}, and Fig.~\ref{fig:epsart}.
%
%\section{Figures and Tables}
%Figures and tables are typically ``floats'' which means that their
%final position is determined by \LaTeX\ while the document is being
%typeset. \LaTeX\ isn't always successful in placing floats
%optimally.
%
%Figures may be inserted by using either the \texttt{graphics} or
%\texttt{graphix} packages. These packages both define the
%\verb+\includegraphics{#1}+ command, but they differ in how optional
%arguments for specifying the orientation, scaling, and translation of the
%figure. Fig.~\ref{fig:epsart} shows a figure that is small enough to
%fit in a single column. It is embedded using the \texttt{figure}
%environment which provides both the caption and the imports the figure
%file.
%\begin{figure}
%\includegraphics{fig_1}% Here is how to import EPS art
%\caption{\label{fig:epsart} A figure caption. The figure captions are
%automatically numbered.}
%\end{figure}
%
%Fig.~\ref{fig:wide} is a figure that is too wide for a single column,
%so instead the \texttt{figure*} environment has been used.
%\begin{figure*}
%\includegraphics{fig_2}% Here is how to import EPS art
%\caption{\label{fig:wide}Use the figure* environment to get a wide
%figure that spans the page in \texttt{twocolumn} formatting.}
%\end{figure*}
%
%The heart of any table is the \texttt{tabular} environment which gives
%the rows of the tables. Each row consists of column entries separated
%by \verb+&+'s and terminates with \textbackslash\textbackslash. The
%required argument for the \texttt{tabular} environment
%specifies how data are displayed in the columns. For instance, entries
%may be centered, left-justified, right-justified, aligned on a decimal
%point. Extra column-spacing may be be specified as well, although
%REV\TeX~4 sets this spacing so that the columns fill the width of the
%table. Horizontal rules are typeset using the \verb+\hline+
%command. The doubled (or Scotch) rules that appear at the top and
%bottom of a table can be achieved enclosing the \texttt{tabular}
%environment within a \texttt{ruledtabular} environment. Rows whose
%columns span multiple columns can be typeset using the
%\verb+\multicolumn{#1}{#2}{#3}+ command (for example, see the first
%row of Table~\ref{tab:table3}).
%
%Tables~\ref{tab:table1}-\ref{tab:table4} show various effects. Tables
%that fit in a narrow column are contained in a \texttt{table}
%environment. Table~\ref{tab:table3} is a wide table set with the
%\texttt{table*} environment. Long tables may need to break across
%pages. The most straightforward way to accomplish this is to specify
%the \verb+[H]+ float placement on the \texttt{table} or
%\texttt{table*} environment. However, the standard \LaTeXe\ package
%\texttt{longtable} will give more control over how tables break and
%will allow headers and footers to be specified for each page of the
%table. A simple example of the use of \texttt{longtable} can be found
%in the file \texttt{summary.tex} that is included with the REV\TeX~4
%distribution.
%
%There are two methods for setting footnotes within a table (these
%footnotes will be displayed directly below the table rather than at
%the bottom of the page or in the bibliography). The easiest
%and preferred method is just to use the \verb+\footnote{#1}+
%command. This will automatically enumerate the footnotes with
%lowercase roman letters. However, it is sometimes necessary to have
%multiple entries in the table share the same footnote. In this case,
%there is no choice but to manually create the footnotes using
%\verb+\footnotemark[#1]+ and \verb+\footnotetext[#1]{#2}+.
%\texttt{\#1} is a numeric value. Each time the same value for
%\texttt{\#1} is used, the same mark is produced in the table. The
%\verb+\footnotetext[#1]{#2}+ commands are placed after the \texttt{tabular}
%environment. Examine the \LaTeX\ source and output for
%Tables~\ref{tab:table1} and \ref{tab:table2} for examples.
%
%\begin{table}
%\caption{\label{tab:table1}This is a narrow table which fits into a
%narrow column when using \texttt{twocolumn} formatting. Note that
%REV\TeX~4 adjusts the intercolumn spacing so that the table fills the
%entire width of the column. Table captions are numbered
%automatically. This table illustrates left-aligned, centered, and
%right-aligned columns.  }
%\begin{ruledtabular}
%\begin{tabular}{lcr}
%Left\footnote{Note a.}&Centered\footnote{Note b.}&Right\\
%\hline
%1 & 2 & 3\\
%10 & 20 & 30\\
%100 & 200 & 300\\
%\end{tabular}
%\end{ruledtabular}
%\end{table}
%
%\begin{table}
%\caption{\label{tab:table2}A table with more columns still fits
%properly in a column. Note that several entries share the same
%footnote. Inspect the \LaTeX\ input for this table to see
%exactly how it is done.}
%\begin{ruledtabular}
%\begin{tabular}{cccccccc}
% &$r_c$ (\AA)&$r_0$ (\AA)&$\kappa r_0$&
% &$r_c$ (\AA) &$r_0$ (\AA)&$\kappa r_0$\\
%\hline
%Cu& 0.800 & 14.10 & 2.550 &Sn\footnotemark[1]
%& 0.680 & 1.870 & 3.700 \\
%Ag& 0.990 & 15.90 & 2.710 &Pb\footnotemark[2]
%& 0.450 & 1.930 & 3.760 \\
%Au& 1.150 & 15.90 & 2.710 &Ca\footnotemark[3]
%& 0.750 & 2.170 & 3.560 \\
%Mg& 0.490 & 17.60 & 3.200 &Sr\footnotemark[4]
%& 0.900 & 2.370 & 3.720 \\
%Zn& 0.300 & 15.20 & 2.970 &Li\footnotemark[2]
%& 0.380 & 1.730 & 2.830 \\
%Cd& 0.530 & 17.10 & 3.160 &Na\footnotemark[5]
%& 0.760 & 2.110 & 3.120 \\
%Hg& 0.550 & 17.80 & 3.220 &K\footnotemark[5]
%&  1.120 & 2.620 & 3.480 \\
%Al& 0.230 & 15.80 & 3.240 &Rb\footnotemark[3]
%& 1.330 & 2.800 & 3.590 \\
%Ga& 0.310 & 16.70 & 3.330 &Cs\footnotemark[4]
%& 1.420 & 3.030 & 3.740 \\
%In& 0.460 & 18.40 & 3.500 &Ba\footnotemark[5]
%& 0.960 & 2.460 & 3.780 \\
%Tl& 0.480 & 18.90 & 3.550 & & & & \\
%\end{tabular}
%\end{ruledtabular}
%\footnotetext[1]{Here's the first, from Ref.~\onlinecite{feyn54}.}
%\footnotetext[2]{Here's the second.}
%\footnotetext[3]{Here's the third.}
%\footnotetext[4]{Here's the fourth.}
%\footnotetext[5]{And etc.}
%\end{table}
%
\begin{table*}
\caption{\label{tab:table1}Summary of properties of various domestic airline networks.}
\begin{ruledtabular}
\begin{tabular}{ccccc}
 &\multicolumn{2}{c}{$D_{4h}^1$}&\multicolumn{2}{c}{$D_{4h}^5$}\\
 Ion&1st alternative&2nd alternative&lst alternative
&2nd alternative\\ \hline
 K&$(2e)+(2f)$&$(4i)$ &$(2c)+(2d)$&$(4f)$ \\
 Mn&$(2g)$\footnote{The $z$ parameter of these positions is $z\sim\frac{1}{4}$.}
 &$(a)+(b)+(c)+(d)$&$(4e)$&$(2a)+(2b)$\\
 Cl&$(a)+(b)+(c)+(d)$&$(2g)$\footnotemark[1]
 &$(4e)^{\text{a}}$\\
 He&$(8r)^{\text{a}}$&$(4j)^{\text{a}}$&$(4g)^{\text{a}}$\\
 Ag& &$(4k)^{\text{a}}$& &$(4h)^{\text{a}}$\\
\end{tabular}
\end{ruledtabular}
\end{table*}
%
%\begin{table}
%\caption{\label{tab:table4}Numbers in columns Three--Five have been
%aligned by using the ``d'' column specifier (requires the
%\texttt{dcolumn} package). Non-numeric entries (those entries without
%a ``.'') in a ``d'' column are aligned on the decimal point. Use the
%``D'' specifier for more complex layouts. }
%\begin{ruledtabular}
%\begin{tabular}{ccddd}
%One&Two&\mbox{Three}&\mbox{Four}&\mbox{Five}\\
%\hline
%one&two&\mbox{three}&\mbox{four}&\mbox{five}\\
%He&2& 2.77234 & 45672. & 0.69 \\
%C\footnote{Some tables require footnotes.}
%  &C\footnote{Some tables need more than one footnote.}
%  & 12537.64 & 37.66345 & 86.37 \\
%\end{tabular}
%\end{ruledtabular}
%\end{table}
%
%\textit{Physical Review} style requires that the initial citation of
%figures or tables be in numerical order in text, so don't cite
%Fig.~\ref{fig:wide} until Fig.~\ref{fig:epsart} has been cited.
%
%\begin{acknowledgments}
%We wish to acknowledge the support of the author community in using
%REV\TeX{}, offering suggestions and encouragement, testing new versions,
%\dots.
%\end{acknowledgments}
%
%\appendix
%
%\section{Appendixes}
%
%To start the appendixes, use the \verb+\appendix+ command.
%This signals that all following section commands refer to appendixes
%instead of regular sections. Therefore, the \verb+\appendix+ command
%should be used only once---to setup the section commands to act as
%appendixes. Thereafter normal section commands are used. The heading
%for a section can be left empty. For example,
%\begin{verbatim}
%\appendix
%\section{}
%\end{verbatim}
%will produce an appendix heading that says ``APPENDIX A'' and
%\begin{verbatim}
%\appendix
%\section{Background}
%\end{verbatim}
%will produce an appendix heading that says ``APPENDIX A: BACKGROUND''
%(note that the colon is set automatically).
%
%If there is only one appendix, then the letter ``A'' should not
%appear. This is suppressed by using the star version of the appendix
%command (\verb+\appendix*+ in the place of \verb+\appendix+).
%
%\section{A little more on appendixes}
%
%Observe that this appendix was started by using
%\begin{verbatim}
%\section{A little more on appendixes}
%\end{verbatim}
%
%Note the equation number in an appendix:
%\begin{equation}
%E=mc^2.
%\end{equation}
%
%\subsection{\label{app:subsec}A subsection in an appendix}
%
%You can use a subsection or subsubsection in an appendix. Note the
%numbering: we are now in Appendix \ref{app:subsec}.
%
%Note the equation numbers in this appendix, produced with the
%subequations environment:
%\begin{subequations}
%\begin{eqnarray}
%E&=&mc, \label{appa}
%\\
%E&=&mc^2, \label{appb}
%\\
%E&\agt& mc^3. \label{appc}
%\end{eqnarray}
%\end{subequations}
%They turn out to be Eqs.~(\ref{appa}), (\ref{appb}), and (\ref{appc}).
%\newpage %Just because of unusual number of tables stacked at end

%\bibliography{apssamp}% Produces the bibliography via BibTeX.

\section{Conclusion}
The conclusion goes here.

\section*{Acknowledgment}
The authors would like to thank...
\begin{thebibliography}{1}
\bibitem{albert02}
R. Albert and A.-L. Barab\'{a}si. \textit{Statistical mechanics of complex networks}. Review of Modern Physics, 74, pp. 47-97, 2002.

\bibitem{barabasi}
A.-L. Barab\'{a}si. \textit{Scale-Free Networks}. Scientific American, 288, pp. 60-69, May 2003.

\bibitem{strogatz}
S.H. Strogatz. \textit{Exploring complex networks}. Nature, Vol. 410, Issue 6825, pp. 268-276, 2001.

\bibitem{barabasi99}
R. Albert, H. Jeong and A.-L. Barab\'{a}si. \textit{Diameter of the world-wide web}. Nature, Vol. 401, Issue 6749, pp. 130-131, 1999.

\bibitem{watts}
D.J. Watts and S.H. Strogatz. \textit{Collective dynamics of 'small-world' networks}. Nature, Vol. 393, Issue 6684, pp. 440-442,v 1998.

\bibitem{kleinberg}
J.M. Kleinberg. \textit{Navigation in a small world}. Nature, Vol. 406, Issue 6798, pp. 845, 2000.

\bibitem{newman}
M.E.J Newman. \textit{Modularity and community structure in networks}. In proc. National Academy of Sciences, Vol. 103, No. 23, pp. 8577-8582, 2006.

\bibitem{guimera07}
R. Guimera, M. Sales-Pardo and L.A.N. Amaral. A network-based method for target selection in metabolic networks. Bioinformatics 23, 1616-1622 (2007).

\bibitem{amaral08}
Smart, AG, Amaral, LAN, & Ottino, JM. Cascading failure and robustness in metabolic networks. Proc. Natl. Acad. Sci. U. S. A. 105, 13223-13228 (2008).

\bibitem{guimera04}
Modeling the worldwide airport network.

\bibitem{amaral05}
Airline network.

\end{thebibliography}

\end{document}
%
% ****** End of file apssamp.tex ******
